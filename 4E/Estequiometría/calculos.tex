% Options for packages loaded elsewhere
\PassOptionsToPackage{unicode}{hyperref}
\PassOptionsToPackage{hyphens}{url}
%
\documentclass[
]{article}
\usepackage{lmodern}
\usepackage{amsmath}
\usepackage{ifxetex,ifluatex}
\ifnum 0\ifxetex 1\fi\ifluatex 1\fi=0 % if pdftex
  \usepackage[T1]{fontenc}
  \usepackage[utf8]{inputenc}
  \usepackage{textcomp} % provide euro and other symbols
  \usepackage{amssymb}
\else % if luatex or xetex
  \usepackage{unicode-math}
  \defaultfontfeatures{Scale=MatchLowercase}
  \defaultfontfeatures[\rmfamily]{Ligatures=TeX,Scale=1}
\fi
% Use upquote if available, for straight quotes in verbatim environments
\IfFileExists{upquote.sty}{\usepackage{upquote}}{}
\IfFileExists{microtype.sty}{% use microtype if available
  \usepackage[]{microtype}
  \UseMicrotypeSet[protrusion]{basicmath} % disable protrusion for tt fonts
}{}
\makeatletter
\@ifundefined{KOMAClassName}{% if non-KOMA class
  \IfFileExists{parskip.sty}{%
    \usepackage{parskip}
  }{% else
    \setlength{\parindent}{0pt}
    \setlength{\parskip}{6pt plus 2pt minus 1pt}}
}{% if KOMA class
  \KOMAoptions{parskip=half}}
\makeatother
\usepackage{xcolor}
\IfFileExists{xurl.sty}{\usepackage{xurl}}{} % add URL line breaks if available
\IfFileExists{bookmark.sty}{\usepackage{bookmark}}{\usepackage{hyperref}}
\hypersetup{
  hidelinks,
  pdfcreator={LaTeX via pandoc}}
\urlstyle{same} % disable monospaced font for URLs
\setlength{\emergencystretch}{3em} % prevent overfull lines
\providecommand{\tightlist}{%
  \setlength{\itemsep}{0pt}\setlength{\parskip}{0pt}}
\setcounter{secnumdepth}{-\maxdimen} % remove section numbering
\ifluatex
  \usepackage{selnolig}  % disable illegal ligatures
\fi

\author{}
\date{}


% MODIFICATIONS MADE BY ME


\usepackage{siunitx}
\usepackage{chemfig, chemformula}
  \setchemfig{atom sep=2em}
\usepackage[no-files]{xsim}
  \loadxsimstyle{ged}
  \xsimsetup{
    % path                  = {xsim},
    exercise/template     = {gedmargin},
    exercise/name         = {},
    exercise/print        = {true},
    solution/template     = {gedsolution},
    solution/name         = {Solución:},
    solution/print        = {true},
    % exercise/within       = section,
    % exercise/the-counter  = \thesection.\arabic{exercise},
  }

% END MODIFICATIONS

\begin{document}

\hypertarget{cuxe1lculos-con-ecuaciones-quuxedmicas}{%
\section{Cálculos con ecuaciones
químicas}\label{cuxe1lculos-con-ecuaciones-quuxedmicas}}

\hypertarget{reacciones-con-gases-y-suxf3lidos}{%
\subsection{Reacciones con gases y
sólidos}\label{reacciones-con-gases-y-suxf3lidos}}

Cuando sea necesario, consulta las masas atómicas de los elementos que
necesites en la tabla periódica.

\begin{exercise}Indica qué masa de magnesio reaccionará completamente
con 32 g de azufre sabiendo que estas dos sustancias reaccionan según la
siguiente ecuación: \[\ch{Mg(s) + S(s) -> MgS(s)}\]\end{exercise}

\begin{solution}24,3 g.\end{solution}

\begin{exercise}El trióxido de azufre, SO\textsubscript{3}, se obtiene a
partir del dióxido de azufre, SO\textsubscript{2}, de acuerdo con esta
ecuación: \[\ch{SO2(g) + O2(g) -> SO3(g)}\]

\begin{enumerate}
\def\labelenumi{\alph{enumi})}
\tightlist
\item
  Ajusta la reacción química.
\item
  Calcula la masa de SO\textsubscript{2} que se requiere para que
  reaccione completamente con 32 g de O\textsubscript{2} .
\item
  Determina la masa de trióxido de azufre que se obtiene en el caso
  anterior.
\item
  Halla el volumen de SO\textsubscript{3} que se obtiene si reaccionan
  completamente 22,4 L de SO\textsubscript{2} a 0 °C y 1 atm.
\end{enumerate}

\end{exercise}

\begin{solution}a) \(\ch{SO2(g) + 1/2 O2(g) -> SO3(g)}\); b) 128 g; c)
160 g; d) 22,4 L\end{solution}

\begin{exercise}La ecuación química ajustada de la reacción de oxidación
del hierro es: \[\ch{4 Fe(s) + 3 O2(g) -> 2 Fe2O3(s)}\]

¿Qué masa de óxido de hierro(III) se produce al oxidar completamente 112
g de hierro?\end{exercise}

\begin{solution}160 g\end{solution}

\begin{exercise}La ecuación química ajustada de la reacción de
combustión del batano es:
\[\ch{2 C4H10(g) + 13 O2(g) -> 8 CO2(g) + 10 H2O(g)}\] ¿Qué volumen de
oxígeno debe utilizarse para la combustión completa de 1 kg de
butano?\end{exercise}

\begin{solution}2510 L\end{solution}

\begin{exercise}El carbonato de calcio, CaCO\textsubscript{3}, se
descompone a elevada temperatura en óxido de calcio, CaO, y dióxido de
carbono, CO\textsubscript{2}

\begin{enumerate}
\def\labelenumi{\alph{enumi})}
\tightlist
\item
  Escribe la ecuación química ajustada.
\item
  Calcula la masa de carbonato de calcio necesaria para obtener 1000 kg
  de óxido de calcio.
\item
  Calcula el volumen de dióxido de carbono que se desprende en
  condiciones normales de presión y temperatura, según el apartado
  anterior.
\item
  Calcula la masa de carbonato de calcio necesaria para obtener 140 kg
  de óxido de calcio.
\item
  Calcula la cantidad de óxido de calcio que se obtiene e la
  descomposición de 100 kg de carbonato de calcio.
\end{enumerate}

\end{exercise}

\begin{solution}a) \(\ch{CaCO3(s) -> CaO(s) + CO2(g)}\); b) 1786 kg; c)
\(\SI{4e5}{\L}\);

\begin{enumerate}
\def\labelenumi{\alph{enumi})}
\setcounter{enumi}{3}
\tightlist
\item
  250 kg; e) 56 kg
\end{enumerate}

\end{solution}

\begin{exercise}Calcula los volúmenes, medidos en condiciones normales,
de hidrógeno y oxígeno que se obtienen en la descomposición de 180 g de
agua.\end{exercise}

\begin{solution}224 L de H\textsubscript{2} y 112 L de
O\textsubscript{2}\end{solution}

\begin{exercise}Todos los hidrocarburos, al quemarse al aire, producen
dióxido de carbono y agua. Calcula el volumen de oxígeno necesario,
medido en condiciones normales de presión y temperatura, para la
combustión completa del gas butano contenido en una bombona de 12,5 kg.
La ecuación de la reacción es: \$\$\textbackslash ch\{2 C4H10(g) + 13
O2(g) -\textgreater{} 8 CO2(g) + 1O H2O(g)\end{exercise}

\begin{solution}31 380 L\end{solution}

\begin{exercise}Calcula el volumen de hidrógeno que se desprende al
hacer reaccionar 6,54 g de zinc con la cantidad suficiente de ácido
clorhídrico, a O °C y 1 atm. La ecuación química (no ajustada) de esta
reacción es: \[\ch{Zn(s) + HCl(aq) -> ZnCl2(aq) + H2(g)}\]\end{exercise}

\begin{solution}2,24 L\end{solution}

\begin{exercise}El sodio reacciona violentamente con el agua: se
desprende hidrógeno gas y se forma hidróxido de sodio. Si reaccionan 1 g
de sodio con la cantidad necesaria de agua, calcula la masa de NaOH
producido según la ecuación:
\[\ch{2 Na(s) + 2 H2O(l) -> 2 NaOH(aq) + H2(g)}\]\end{exercise}

\begin{solution}1,74 g\end{solution}

\begin{exercise}En la reacción del aluminio con el oxígeno para dar
óxido de aluminio se han utilizado 81 g de aluminio.

\begin{enumerate}
\def\labelenumi{\alph{enumi})}
\tightlist
\item
  Escribe la ecuación química ajustada de la reacción.
\item
  ¿Qué volumen de oxígeno, a 0 °C y 1 atm, es necesario para oxidar por
  completo el aluminio?
\item
  ¿Qué cantidad de óxido de aluminio se obtiene?
\end{enumerate}

\end{exercise}

\begin{solution}b) 50,4 L; c) 1 53 g\end{solution}

\begin{exercise}El amoniaco se descompone en nitrógeno e hidrógeno,
ambos en estado gaseoso.

\begin{enumerate}
\def\labelenumi{\alph{enumi})}
\tightlist
\item
  Escribe la ecuación química ajustada de la reacción.
\item
  Calcula el volumen de H\textsubscript{2} que se desprende, a 0 °C y 1
  atm, en la descomposición de 68 g de NH\textsubscript{3}.
\item
  Calcula el volumen de nitrógeno que se desprende en estas mismas
  condiciones.
\end{enumerate}

\end{exercise}

\begin{solution}a) \(\ch{2 NH3(g) -> N2(g) + 3 H2(g)}\); b) 134,4 L; c)
44,8 L\end{solution}

\begin{exercise}Cuando el zinc reacciona con el ácido clorhídrico se
obtiene cloruro de zinc y se desprende hidrógeno:

\begin{enumerate}
\def\labelenumi{\alph{enumi})}
\tightlist
\item
  Escribe y equilibra la ecuación química.
\item
  ¿Qué cantidad de HCl se necesita para que reaccione con 1 mol de zinc?
\item
  ¿Qué volumen de hidrógeno se producirá a partir de la reacción
  completa de 20 g de zinc?
\end{enumerate}

\end{exercise}

\begin{solution}a) \(\ch{Zn(s) + HCl(aq) -> ZnCl2(s) + H2(g)}\); b) 2
mol ; c) 6,72 L\end{solution}

\begin{exercise}El carbonato de cobre(II) se descompone, por acción del
calor, de acuerdo con la sigu iente ecuación química :
\[\ch{CuCO3(s) -> CuO(s) + CO2(g)}\]

\begin{enumerate}
\def\labelenumi{\alph{enumi})}
\tightlist
\item
  Si se descomponen 31 g de carbonato de cobre, ¿ qué masa de dióxido de
  carbono se obtiene?
\item
  ¿Qué volumen de este gas se obtiene, medido a 0 °C y 1 atm?
\item
  ¿Qué cantidad de óxido de cobre resulta?
\end{enumerate}

\end{exercise}

\begin{solution}a) 11 g; b) 5,6 L; c) 0,25 mol.\end{solution}

\hypertarget{reacciones-en-disoluciuxf3n}{%
\subsection{Reacciones en
disolución}\label{reacciones-en-disoluciuxf3n}}

\begin{exercise}El nitrato de plata, AgNO\textsubscript{3}, en
disolución acuosa reacciona con el sulfuro de sodio,
Na\textsubscript{2}S, en disolución acuosa para dar sulfuro de plata más
nitrato de sodio. Calcula el volumen de disolución 0,1 M de nitrato de
plata necesario para reaccionar exactamente con 200 mL de disolución de
sulfuro de sodio 0,1 M.

Solución: 400 mL\end{exercise}

\begin{exercise}El zinc reacciona con el ácido clorhídrico produciendo
cloruro de zinc y desprendiendo gas hidrógeno. Halla la cantidad de
zinc, en mol, que se necesita para reaccionar completamente con 100 mL
de una disolución de ácido clorhídrico 0,5 M.

Solución: 0,025 mol\end{exercise}

\begin{exercise}El etanol reacciona con el ácido etanoico (ácido
acético) para dar etanoato de etilo (acetato de etilo) y agua. ¿Qué
volumen de una disolución 0,2 M de etanol se necesita para reaccionar
exactamente con 150 mL de una disolución de ácido etanoico 0,4 M?

Solución: 300 mL\end{exercise}

\begin{exercise}El ácido nítrico reacciona con el hidróxido de amonio,
NH\textsubscript{4}OH, para dar nitrato de amonio y agua. Calcula el
volumen de ácido nítrico 0,01 M necesario para que reaccione
completamente con 250 mL de hidróxido de amonio 0,05 M.

Solución: 1,25 L\end{exercise}

\end{document}

% Options for packages loaded elsewhere
\PassOptionsToPackage{unicode}{hyperref}
\PassOptionsToPackage{hyphens}{url}
%
\documentclass[
]{article}
\usepackage{lmodern}
\usepackage{amsmath}
\usepackage{ifxetex,ifluatex}
\ifnum 0\ifxetex 1\fi\ifluatex 1\fi=0 % if pdftex
  \usepackage[T1]{fontenc}
  \usepackage[utf8]{inputenc}
  \usepackage{textcomp} % provide euro and other symbols
  \usepackage{amssymb}
\else % if luatex or xetex
  \usepackage{unicode-math}
  \defaultfontfeatures{Scale=MatchLowercase}
  \defaultfontfeatures[\rmfamily]{Ligatures=TeX,Scale=1}
\fi
% Use upquote if available, for straight quotes in verbatim environments
\IfFileExists{upquote.sty}{\usepackage{upquote}}{}
\IfFileExists{microtype.sty}{% use microtype if available
  \usepackage[]{microtype}
  \UseMicrotypeSet[protrusion]{basicmath} % disable protrusion for tt fonts
}{}
\makeatletter
\@ifundefined{KOMAClassName}{% if non-KOMA class
  \IfFileExists{parskip.sty}{%
    \usepackage{parskip}
  }{% else
    \setlength{\parindent}{0pt}
    \setlength{\parskip}{6pt plus 2pt minus 1pt}}
}{% if KOMA class
  \KOMAoptions{parskip=half}}
\makeatother
\usepackage{xcolor}
\IfFileExists{xurl.sty}{\usepackage{xurl}}{} % add URL line breaks if available
\IfFileExists{bookmark.sty}{\usepackage{bookmark}}{\usepackage{hyperref}}
\hypersetup{
  hidelinks,
  pdfcreator={LaTeX via pandoc}}
\urlstyle{same} % disable monospaced font for URLs
\setlength{\emergencystretch}{3em} % prevent overfull lines
\providecommand{\tightlist}{%
  \setlength{\itemsep}{0pt}\setlength{\parskip}{0pt}}
\setcounter{secnumdepth}{-\maxdimen} % remove section numbering
\ifluatex
  \usepackage{selnolig}  % disable illegal ligatures
\fi

\author{}
\date{}


% MODIFICATIONS MADE BY ME
\usepackage{siunitx}
\usepackage{chemfig, chemformula}
  \setchemfig{atom sep=2em}
\usepackage[no-files]{xsim}
  \loadxsimstyle{layouts,ged}
  \xsimsetup{
    % path                  = {xsim},
    exercise/template     = {gedmargin},
    exercise/name         = {},
    exercise/print        = {true},
    solution/template     = {gedsolution},
    solution/name         = {Solución: },
    solution/print        = {true},
    % exercise/within       = section,
    % exercise/the-counter  = \thesection.\arabic{exercise},
  }

\begin{document}

\hypertarget{el-sistema-internacional-de-unidades}{%
\section{El Sistema Internacional de
Unidades}\label{el-sistema-internacional-de-unidades}}

\begin{enumerate}
\def\labelenumi{\arabic{enumi}.}
\item
  Pon dos ejemplos de propiedades que sean magnitudes físicas, y otros
  dos que no lo sean. Para cada una de las primeras, enumera al menos
  tres unidades, una de ellas la del SI.
\item
  Busca varios ejemplos de unidades que no pertenezcan al SI. ¿Con qué
  magnitudes están relacionadas? Indica la equivalencia entre estas
  unidades y las correspondientes del SI.
\item
  ¿Crees que la yarda, definida en su día como unidad de longitud y
  equivalente a 914~mm, y obtenida por la distancia marcada en una vara
  entre la nariz y el dedo pulgar de la mano del rey Enrique I de
  Inglaterra con su brazo estirado, sería hoy un procedimiento adecuado
  para establecer una unidad de longitud?
\item
  Las unidades del SI han sufrido cambios en su definición a lo largo de
  la historia. Por ejemplo, el metro se definió en 1790 como la
  diezmillonésima parte del cuadrante del meridiano terrestre que pasa
  por París. En 1889 fue la distancia entre dos marcas en una barra de
  aleación de platino-iridio que se guarda en Sévres. La definición
  actual es de 1983. ¿A qué se deben estos cambios?
\end{enumerate}

\hypertarget{analisis-dimensional}{%
\section{Análisis dimensional}\label{analisis-dimensional}}

\begin{enumerate}
\def\labelenumi{\arabic{enumi}.}
\item
  La ley de la gravitación universal de Newton establece que la fuerza
  con la que dos cuerpos se atraen es directamente proporcional al
  producto de sus masas e inversamente proporcional al cuadrado de la
  distancia que las separa: \[F = G\frac{m \cdot m'}{r^2}\] donde \(G\)
  es la constante universal de la gravitación. Determina la ecuación
  dimensional de esta constante y, a partir de ella, su unidad del SI.
\item
  La fuerza se puede calcular como el producto de la masa por la
  aceleración. Calcula la ecuación de dimensiones de esta magnitud.
\item
  Comprueba que la siguiente fórmula es homogénea, es decir, que la
  dimensión de todos los términos es una longitud:
  \[s = v_0 \cdot t + \frac{1}{2} a \cdot t^2\]
\item
  Escribe la ecuación de dimensiones para la aceleración y la fuerza.
\item
  Comprueba que la ecuación \(v = \sqrt{2 \cdot g \cdot\ h}\), que
  determina la velocidad de caída libre de un objeto, es homogénea.
\item
  Deduce la ecuación de dimensión de la magnitud física trabajo,
  definida matemáticamente como: \(W = F \cdot \Delta r\), e indica la
  expresión de su unidad, el julio, en función de las unidades
  fundamentales del SI.
\item
  La energía intercambiada en forma de calor por un objeto al
  modificarse su temperatura se determina mediante la expresión:
  \(Q = m \cdot c_e \cdot \Delta T\). Determina la unidad del SI en la
  que se mide la constante calor específico \(c_e\).
\item
  Expresa en la unidad adecuada del Sistema Internacional las magnitudes
  expresadas por las siguientes ecuaciones de dimensión:

  \begin{enumerate}
  \def\labelenumii{\arabic{enumii}.}
  \item
    \(\mathrm{MLT^{-2}}\)
  \item
    \(\mathrm{ML{-3}}\)
  \item
    \(\mathrm{LT^{-1}}\)
  \item
    \(\mathrm{ML^2T^{-2}}\)
  \end{enumerate}
\end{enumerate}

\hypertarget{factores-de-conversion}{%
\section{Factores de conversión}\label{factores-de-conversion}}

\begin{enumerate}
\def\labelenumi{\arabic{enumi}.}
\item
  Expresa las siguientes medidas en el SI, respetando el número de
  cifras significativas que poseen:

  \begin{enumerate}
  \def\labelenumii{\arabic{enumii}.}
  \item
    29~cm
  \item
    100.0~dg
  \item
    144~km/h
  \item
    34.65~dm\textsuperscript{2}
  \end{enumerate}
\item
  Utiliza factores de conversión para realizar los siguientes cambios de
  unidades:

  \begin{enumerate}
  \def\labelenumii{\arabic{enumii}.}
  \item
    324500~mg a Mg
  \item
    3~cg a Gg
  \item
    90~dm a nm
  \item
    86400~ms a h
  \item
    0.35~hL a uL
  \item
    0.335~mL a uL
  \item
    0.00092~hm a pm
  \item
    0.075~hg a ng
  \end{enumerate}
\item
  Utiliza factores de conversión para realizar los siguientes cambios de
  unidades de superficie y volumen:

  \begin{enumerate}
  \def\labelenumii{\arabic{enumii}.}
  \item
    1850.2~cm\textsuperscript{2} a m\textsuperscript{2}
  \item
    0.00245~d\textsuperscript{2}m a um\textsuperscript{2}
  \item
    25680~mm\textsuperscript{2} a h\textsuperscript{2}m
  \item
    2300~pm\textsuperscript{3} a dm\textsuperscript{3}
  \item
    9530~d\textsuperscript{3}m a nm\textsuperscript{3}
  \item
    0.003~mm\textsuperscript{3} a d\textsuperscript{3}m
  \item
    50~m\textsuperscript{3} a L
  \item
    0.3~dm\textsuperscript{3} a nL
  \item
    80~kL a dm\textsuperscript{3}
  \item
    25460~uL a m\textsuperscript{3}
  \end{enumerate}
\end{enumerate}

\begin{enumerate}
\def\labelenumi{\arabic{enumi}.}
\item
  Escribe correctamente los siguentes números en notación científica:

  \begin{enumerate}
  \def\labelenumii{\arabic{enumii}.}
  \item
    340000
  \item
    0.000076
  \item
    750~×~10\textsuperscript{3}
  \item
    13800000
  \item
    0.000005
  \item
    4800000000
  \item
    0.0000173
  \item
    45872300
  \item
    0.0004~×~10\textsuperscript{8}
  \item
    0.091e-9
  \end{enumerate}
\end{enumerate}

\begin{enumerate}
\def\labelenumi{\arabic{enumi}.}
\item
  Utiliza factores de conversión para realizar los siguientes cambios de
  unidades:

  \begin{enumerate}
  \def\labelenumii{\arabic{enumii}.}
  \item
    90~m/s a km/h
  \item
    540~km/h a m/s
  \item
    4.5~×~10\textsuperscript{3}~kg/mm\textsuperscript{3} a dg/mL
  \item
    0.03~×~10\textsuperscript{4}~pm\textsuperscript{3}/ns a kL/ds
  \item
    240~um/min a dm/s
  \item
    80~gmm/s a kgm/h
  \end{enumerate}
\end{enumerate}

\begin{enumerate}
\def\labelenumi{\arabic{enumi}.}
\item
  Utilizando factores de conversión, cambia las siguientes medidas a
  unidades del SI, expresando los resultados en notación científica:

  \begin{enumerate}
  \def\labelenumii{\arabic{enumii}.}
  \item
    934.8~hm/min
  \item
    0.0023~Mg/pLnm\textsuperscript{2}
  \item
    35~cg/L
  \item
    162.3~dgL/nm\textsuperscript{3}
  \item
    0.5~kgm/min
  \item
    7~kgm/ds
  \item
    256~p\textsuperscript{2}mug/h
  \item
    0.03~hL/ms
  \end{enumerate}
\end{enumerate}

\hypertarget{errores-e-incertidumbre}{%
\section{Errores e incertidumbre}\label{errores-e-incertidumbre}}

\begin{enumerate}
\def\labelenumi{\arabic{enumi}.}
\item
  ¿Puede ser una medida precisa pero inexacta? ¿Y lo contrario? Propón
  algún ejemplo para apoyar tu razonamiento.
\item
  Si mides tu altura con una cinta métrica, y el error absoluto de la
  medida es de 1~cm, ¿cuál es el error relativo? Exprésalo mediante
  valor numérico y porcentual.
\item
  Se realizan cinco medidas de la longitud de la mesa del laboratorio
  con una cinta métrica, obteniendo estos valores, en cm: 120.6, 120.4,
  120.5, , \& 120.3. ¿Cuál sería el valor de la medida? ¿Y los errores,
  absoluto y relativo?
\item
  Expresa las siguientes cantidades en notación científica:

  \begin{enumerate}
  \def\labelenumii{\arabic{enumii}.}
  \item
    75600000~g.
  \item
    0.000000025~V.
  \item
    149800000~km.
  \end{enumerate}
\item
  ¿Oué puedes concluir de las diferencias entre las cantidades numéricas
  2.0, 2.00, \& 2.000, procedentes de la medida experimental de una
  magnitud física?
\item
  ¿Sería correcto decir que una medida de tiempo da como resultado
  \(t = \SI{1.35 +- 0.15}{\s}\)? ¿Por qué?
\item
  Medimos la longitud de una mesa y resulta un valor de 98.50~cm.
  ¿Cuántas cifras significativas tiene la medida? Expresa el valor en
  metros y en milímetros indicando, en cada caso, el número de cifras
  significativas.
\item
  ¿Cómo hallarías el volumen exterior de un portalápices cilíndrico si
  dispones de una regla milimetrada? ¿Qué tipo de medida realizas,
  directa o indirecta?
\item
  Se mide la masa y el volumen de un sólido, obteniendo como resultados
  \(m = \SI{46,72}{\g}\) y \(V = \SI{24,5}{\cubic\cm}\). Calcula su
  densidad y escríbela con el número de decimales correcto.
\item
  Calcula la masa molecular del hidrogenosulfato de sodio (\(NaHSO_4\))
  y exprésala de forma correcta.
\item
  ¿Qué incertidumbre y error relativo cometemos al calcular la masa
  molar del NaCl? (Para facilitar el cálculo tomamos como masa atómica
  del cloro 35,5 en vez de su valor real 35,4527, y la del sodio como
  23, en lugar de 22,9898.)
\item
  Suma las magnitudes siguientes:

  \begin{enumerate}
  \def\labelenumii{\arabic{enumii}.}
  \item
    \(m_1 = \SI{5}{\kg}\) y \(m_2 = \SI{6}{\kg}\)
  \item
    \(\vec{v}_1 = 3\vec{i} - 2\vec{j}\) y
    \(\vec{v}_2 = -2\vec{i} + 2\vec{j}\)
  \end{enumerate}
\item
  Al afirmar que el año luz equivale a 9460800000000~km:

  \begin{enumerate}
  \def\labelenumii{\arabic{enumii}.}
  \item
    ¿Con cuántas cifras significativas se está expresando la medida?
  \item
    ¿Cuáles de ellas son cifras exactas y cuál la sometida a error?
  \end{enumerate}
\item
  Dados los siguientes valores:
  \[A = 263,04; \qquad B = 0,00714; \qquad C = 20,4\] Determina los
  resultados de estas operaciones y exprésalos correctamente:

  \begin{enumerate}
  \def\labelenumii{\arabic{enumii}.}
  \item
    \(A \div C\);
  \item
    \(A \cdot B - C\);
  \item
    \(B \cdot C\)
  \end{enumerate}
\item
  Escribe los siguientes datos en notación científica y en unidades del
  SI:

  \begin{enumerate}
  \def\labelenumii{\arabic{enumii}.}
  \item
    Distancia entre la Tierra y el Sol: 150000000~km.
  \item
    Tamaño de un virus: 18.5~um.
  \end{enumerate}
\item
  Se han calculado la longitud de una torre (22.4~m) con un error de
  ~±~20~cm, y la de una mesa de pupitre (0.8~m) con un error de ~±~1~cm.

  \begin{enumerate}
  \def\labelenumii{\arabic{enumii}.}
  \item
    Indica el error absoluto de cada medida.
  \item
    ¿Cuál de las dos medidas es más precisa?
  \end{enumerate}
\item
  ¿Cuál de las dos siguientes medidas es más precisa: la anchura de un
  folio de papel, que es 210~±~1~mm, o la distancia entre dos ciudades,
  que es 225~±~1~km?
\item
  Dada la longitud 3.2~±~0.1~m. Determina la incertidumbre relativa
  porcentual de la medida.
\item
  Señala el número de cifras significativas en las siguientes medidas de
  longitud:

  \begin{enumerate}
  \def\labelenumii{\arabic{enumii}.}
  \item
    1.55~m;
  \item
    9.02~m;
  \item
    0.010~cm;
  \item
    1.00~×~10\textsuperscript{3}~cm;
  \item
    2500~cm;
  \end{enumerate}
\item
  La sensibilidad de una balanza que mide hasta 10~kg es de ~±~10~g,
  mientras otra mide hasta 10~g y tiene una sensibilidad de ~±~1~g.
  ¿Cuáles la mejor balanza de las dos?
\end{enumerate}

\end{document}

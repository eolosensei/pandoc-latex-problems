% Options for packages loaded elsewhere
\PassOptionsToPackage{unicode}{hyperref}
\PassOptionsToPackage{hyphens}{url}
%
\documentclass[
]{article}
\usepackage{lmodern}
\usepackage{amsmath}
\usepackage{ifxetex,ifluatex}
\ifnum 0\ifxetex 1\fi\ifluatex 1\fi=0 % if pdftex
  \usepackage[T1]{fontenc}
  \usepackage[utf8]{inputenc}
  \usepackage{textcomp} % provide euro and other symbols
  \usepackage{amssymb}
\else % if luatex or xetex
  \usepackage{unicode-math}
  \defaultfontfeatures{Scale=MatchLowercase}
  \defaultfontfeatures[\rmfamily]{Ligatures=TeX,Scale=1}
\fi
% Use upquote if available, for straight quotes in verbatim environments
\IfFileExists{upquote.sty}{\usepackage{upquote}}{}
\IfFileExists{microtype.sty}{% use microtype if available
  \usepackage[]{microtype}
  \UseMicrotypeSet[protrusion]{basicmath} % disable protrusion for tt fonts
}{}
\makeatletter
\@ifundefined{KOMAClassName}{% if non-KOMA class
  \IfFileExists{parskip.sty}{%
    \usepackage{parskip}
  }{% else
    \setlength{\parindent}{0pt}
    \setlength{\parskip}{6pt plus 2pt minus 1pt}}
}{% if KOMA class
  \KOMAoptions{parskip=half}}
\makeatother
\usepackage{xcolor}
\IfFileExists{xurl.sty}{\usepackage{xurl}}{} % add URL line breaks if available
\IfFileExists{bookmark.sty}{\usepackage{bookmark}}{\usepackage{hyperref}}
\hypersetup{
  hidelinks,
  pdfcreator={LaTeX via pandoc}}
\urlstyle{same} % disable monospaced font for URLs
\setlength{\emergencystretch}{3em} % prevent overfull lines
\providecommand{\tightlist}{%
  \setlength{\itemsep}{0pt}\setlength{\parskip}{0pt}}
\setcounter{secnumdepth}{-\maxdimen} % remove section numbering
\ifluatex
  \usepackage{selnolig}  % disable illegal ligatures
\fi

\author{}
\date{}


% MODIFICATIONS MADE BY ME


\usepackage{siunitx}
\usepackage{chemfig, chemformula}
  \setchemfig{atom sep=2em}
\usepackage[no-files]{xsim}
  \loadxsimstyle{ged}
  \xsimsetup{
    % path                  = {xsim},
    exercise/template     = {gedmargin},
    exercise/name         = {},
    exercise/print        = {true},
    solution/template     = {gedsolution},
    solution/name         = {Solución:},
    solution/print        = {true},
    % exercise/within       = section,
    % exercise/the-counter  = \thesection.\arabic{exercise},
  }

% END MODIFICATIONS

\begin{document}

\hypertarget{problemas-de-estequiometruxeda}{%
\section{Problemas de
estequiometría}\label{problemas-de-estequiometruxeda}}

\hypertarget{ajuste-de-reacciones-quuxedmicas}{%
\subsection{Ajuste de reacciones
químicas}\label{ajuste-de-reacciones-quuxedmicas}}

\begin{exercise}Ajusta las siguientes reacciones químicas:

\begin{enumerate}
\def\labelenumi{\alph{enumi})}
\item
  \ch{Ca(OH)2 + HNO3 -> Ca(NO3)2 + H2O}
\item
  \ch{HBF4 + H2O -> H3BO3 + HF}
\item
  \ch{C4H10 + O2 -> CO2 + H2O}
\item
  \ch{Cu(NO3)2 -> CuO + NO2 + O2}
\item
  \ch{CO2 + H2O + CaSiO3 -> SiO2 + Ca(HCO3)2}
\item
  \ch{BCl3 + P4 + H2 -> BP + HCl}
\item
  \ch{HClO4 + P4O10 -> H3PO4 + Cl2O7}
\item
  \ch{KI + Pb(NO3)2 -> KNO3 + PbI2}
\end{enumerate}

\end{exercise}

\hypertarget{reactivo-limitante}{%
\subsection{Reactivo limitante}\label{reactivo-limitante}}

\begin{exercise}[tags=OXF15] Se hacen reaccionar 20 g de \ch{H2} con 100
g de \ch{N2}. Calcula la masa de \ch{NH3} que se obtendrá.\end{exercise}

\begin{solution}113,3 g\end{solution}

\end{document}

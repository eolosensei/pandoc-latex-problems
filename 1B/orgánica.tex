% Options for packages loaded elsewhere
\PassOptionsToPackage{unicode}{hyperref}
\PassOptionsToPackage{hyphens}{url}
%
\documentclass[
]{article}
\usepackage{lmodern}
\usepackage{amsmath}
\usepackage{ifxetex,ifluatex}
\ifnum 0\ifxetex 1\fi\ifluatex 1\fi=0 % if pdftex
  \usepackage[T1]{fontenc}
  \usepackage[utf8]{inputenc}
  \usepackage{textcomp} % provide euro and other symbols
  \usepackage{amssymb}
\else % if luatex or xetex
  \usepackage{unicode-math}
  \defaultfontfeatures{Scale=MatchLowercase}
  \defaultfontfeatures[\rmfamily]{Ligatures=TeX,Scale=1}
\fi
% Use upquote if available, for straight quotes in verbatim environments
\IfFileExists{upquote.sty}{\usepackage{upquote}}{}
\IfFileExists{microtype.sty}{% use microtype if available
  \usepackage[]{microtype}
  \UseMicrotypeSet[protrusion]{basicmath} % disable protrusion for tt fonts
}{}
\makeatletter
\@ifundefined{KOMAClassName}{% if non-KOMA class
  \IfFileExists{parskip.sty}{%
    \usepackage{parskip}
  }{% else
    \setlength{\parindent}{0pt}
    \setlength{\parskip}{6pt plus 2pt minus 1pt}}
}{% if KOMA class
  \KOMAoptions{parskip=half}}
\makeatother
\usepackage{xcolor}
\IfFileExists{xurl.sty}{\usepackage{xurl}}{} % add URL line breaks if available
\IfFileExists{bookmark.sty}{\usepackage{bookmark}}{\usepackage{hyperref}}
\hypersetup{
  pdftitle={Química orgánica},
  hidelinks,
  pdfcreator={LaTeX via pandoc}}
\urlstyle{same} % disable monospaced font for URLs
\usepackage[a5paper,twoside,bindingoffset=10mm]{geometry}
\setlength{\emergencystretch}{3em} % prevent overfull lines
\providecommand{\tightlist}{%
  \setlength{\itemsep}{0pt}\setlength{\parskip}{0pt}}
\setcounter{secnumdepth}{-\maxdimen} % remove section numbering
\ifluatex
  \usepackage{selnolig}  % disable illegal ligatures
\fi

\title{Química orgánica}
\author{}
\date{}


% MODIFICATIONS MADE BY ME


\usepackage{siunitx}
\usepackage{chemfig, chemformula}
  \setchemfig{atom sep=2em}
\usepackage[no-files]{xsim}
  \loadxsimstyle{ged}
  \xsimsetup{
    % path                  = {xsim},
    exercise/template     = {gedmargin},
    exercise/name         = {},
    exercise/print        = {true},
    solution/template     = {gedsolution},
    solution/name         = {Solución:},
    solution/print        = {true},
    % exercise/within       = section,
    % exercise/the-counter  = \thesection.\arabic{exercise},
  }

% END MODIFICATIONS

\begin{document}
\maketitle

\hypertarget{quuxedmica-orguxe1nica}{%
\subsection{Química orgánica}\label{quuxedmica-orguxe1nica}}

\begin{exercise}[tags=OXF15] Formula los siguientes compuestos:

\begin{enumerate}
\def\labelenumi{\alph{enumi})}
\tightlist
\item
  2-metilbutan-2-ol
\item
  Etilfeniléter
\item
  Ciclohexano-1,4-diona
\item
  4-etil-4-metilheptano
\item
  Octa-2,4-dieno
\item
  3-etilocta-7,5-diino
\item
  Pent-3-en-1-ino
\item
  2-etil-3-metilhepta-1,3-dien-6-ino
\item
  Ciclohexino
\item
  Ciclopenta-1,3-dieno
\item
  m-dimetilbenceno
\item
  2-metilbutano-1,3-diol
\item
  3-metilpent-2-enal
\item
  4-fenilpentan-2-ona
\item
  3,3-dimetilpentanodiona
\item
  Acido pent-2-enoico
\item
  Ácido pent-2-enodioico
\item
  Acetato de etilo (etanoato de etilo)
\item
  Butanamida
\item
  Benzamida
\item
  Butano-1,4-diamina
\end{enumerate}

\end{exercise}

\begin{solution}

\begin{enumerate}
\def\labelenumi{\alph{enumi})}
\item
  \ch{CH3-C(CH3)OH-CH2-CH3}
\item
  \chemfig{CH_3-CH_2-O-**6(------)}
\item
  \chemfig{O=*6(---(=O)---)}
\item
  \chemfig{CH_3-CH_2-CH_2-C(-[2]CH_2-CH_3)(-[6]CH_3)-CH_2-CH_2-CH_3}
\item
  \ch{CH3-CH=CH-CH=CH-CH2-CH2-CH3}
\item
  \ch{HC+C-CH(CH3)-CH2-C+C-CH2-CH3}
\item
  \ch{HC+CH2-CH=CH-CH3}
\item
  \chemfig{CH_2=C(-[2]CH_2-CH_3)-C(-[6]CH_3)=CH-CH_2-C~CH}
\item
  \chemfig{[:-120]*6(-~----)}
\item
  \chemfig{[:-126]*5(=-=--)}
\item
  \chemfig{[:-120]**6((-CH_3)----(-CH_3)--)}
\item
  \ch{HO-CH2-CH(CH3)-CHOH-CH3}
\item
  \ch{CH3-CH(CH3)-CH=CH-CHO}
\item
  \chemfig{CH_3-C(=[6]O)-CH_2-CH(-[6]**6(------))-CH_3}
\item
  \ch{CH3-CHO-C(CH3)2-CHO-CH3}
\item
  \ch{CH3-CH2-CH=CH-COOH}
\item
  \ch{HOOC-CH=CH-CH2-COOH}
\item
  \ch{CH3-COO-CH2-CH3}
\item
  \ch{CH3-CH2-CH2-CONH2}
\item
  \chemfig{[:-120]**6(-----(-C(=[2]O)-NH_2)-)}
\item
  \ch{H2N-CH-CH2-CH2-CH-NH2}
\end{enumerate}

\end{solution}

\begin{exercise}[tags=OXF15] Nombra los siguientes compuestos:

\begin{enumerate}
\def\labelenumi{\alph{enumi})}
\item
  \chemfig{CH_3-C(-[2]CH_3)(-[6]CH_3)-C(-[2]CH_3)(-[6]CH_3)-CH_2-CH_3}
\item
  \ch{CH2=CH-CH=CH2}
\item
  \ch{HC+CH-C+C-CH3}
\item
  \ch{CH2=CH-CH2-C+C-CH3}
\item
  \chemfig{[:-126]*5((-CH_3)--=-=)}
\item
  \chemfig{*4(-(-CH_3)--(-CH_3)-)}
\item
  \chemfig{[:-120]**6((-OH)------)}
\item
  \chemfig{[:-120]**6(----(-CH_3)--)}
\item
  \chemfig{[:-120]**6(--(-CH_3)--(-CH_3)--)}
\item
  \ch{CHO-C+C-CH2-CHO}
\item
  \ch{CH3-CH=CH-CH(CH3)-COOH}
\item
  \ch{CH3-CH2-COO-CH3}
\item
  \chemfig{CH_3-CH(-[6]**6(------))-COOH}
\item
  \chemfig{NH(-[4]**6(------))-**6(------)}
\item
  \ch{CH3-CO-NH-CH3}
\item
  \ch{CH3-CH2-CO-NH2}
\item
  \ch{CH3-CH2-CH=CH-CO-NH2}
\item
  \chemfig{CH_3-C(-[2]OH)(-[6]CH_3)-C(-[2]OH)(-[6]CH_3)-CH_3}
\item
  \chemfig{CHO-[4]**6(------)}
\item
  \ch{CH3-CO-CH2-CH3}
\item
  \ch{CH3-CH2-CHCl-COOH}
\item
  \chemfig{CH_2(-[6]OH)-CH(-[6]**6(------))-CH_3}
\end{enumerate}

\end{exercise}

\begin{solution}

\begin{enumerate}
\def\labelenumi{\alph{enumi})}
\tightlist
\item
  2,2,3,3-tetrametilpentano
\item
  But-1,3-dieno
\item
  Pent-1,3-diino
\item
  Hexa-1-en-4-ino
\item
  1-metilciclopenta-1,3-dieno
\item
  1,3-metilciclobutano
\item
  Bencenol (fenol)
\item
  Metilbenceno (tolueno)
\item
  1,3-dimetilbenceno (m-metiltolueno)
\item
  Pent-2-inodial
\item
  Ácido 2-metilpent-3-enoico
\item
  Propanoato de metilo
\item
  Ácido 2-fenilpropanoico
\item
  Difenilamina
\item
  N-metiletanamida
\item
  Propanamida
\item
  Pent-2-enamida
\item
  2,3-dimetilbutan-2,3-diol
\item
  Benzaldehído (bencenal, fenilmetanal)
\item
  Butan-2-ona
\item
  Ácido 2-clorobutanoico
\item
  2-fenilpropan-1-ol
\end{enumerate}

\end{solution}

\end{document}

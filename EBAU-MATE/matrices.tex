% Options for packages loaded elsewhere
\PassOptionsToPackage{unicode}{hyperref}
\PassOptionsToPackage{hyphens}{url}
%
\documentclass[
]{article}
\usepackage{lmodern}
\usepackage{amsmath}
\usepackage{ifxetex,ifluatex}
\ifnum 0\ifxetex 1\fi\ifluatex 1\fi=0 % if pdftex
  \usepackage[T1]{fontenc}
  \usepackage[utf8]{inputenc}
  \usepackage{textcomp} % provide euro and other symbols
  \usepackage{amssymb}
\else % if luatex or xetex
  \usepackage{unicode-math}
  \defaultfontfeatures{Scale=MatchLowercase}
  \defaultfontfeatures[\rmfamily]{Ligatures=TeX,Scale=1}
\fi
% Use upquote if available, for straight quotes in verbatim environments
\IfFileExists{upquote.sty}{\usepackage{upquote}}{}
\IfFileExists{microtype.sty}{% use microtype if available
  \usepackage[]{microtype}
  \UseMicrotypeSet[protrusion]{basicmath} % disable protrusion for tt fonts
}{}
\makeatletter
\@ifundefined{KOMAClassName}{% if non-KOMA class
  \IfFileExists{parskip.sty}{%
    \usepackage{parskip}
  }{% else
    \setlength{\parindent}{0pt}
    \setlength{\parskip}{6pt plus 2pt minus 1pt}}
}{% if KOMA class
  \KOMAoptions{parskip=half}}
\makeatother
\usepackage{xcolor}
\IfFileExists{xurl.sty}{\usepackage{xurl}}{} % add URL line breaks if available
\IfFileExists{bookmark.sty}{\usepackage{bookmark}}{\usepackage{hyperref}}
\hypersetup{
  hidelinks,
  pdfcreator={LaTeX via pandoc}}
\urlstyle{same} % disable monospaced font for URLs
\setlength{\emergencystretch}{3em} % prevent overfull lines
\providecommand{\tightlist}{%
  \setlength{\itemsep}{0pt}\setlength{\parskip}{0pt}}
\setcounter{secnumdepth}{-\maxdimen} % remove section numbering
\ifluatex
  \usepackage{selnolig}  % disable illegal ligatures
\fi

\author{}
\date{}


% MODIFICATIONS MADE BY ME


\usepackage{siunitx}
\usepackage{chemfig, chemformula}
  \setchemfig{atom sep=2em}
\usepackage[no-files]{xsim}
  \loadxsimstyle{ged}
  \xsimsetup{
    % path                  = {xsim},
    exercise/template     = {gedmargin},
    exercise/name         = {},
    exercise/print        = {true},
    solution/template     = {gedsolution},
    solution/name         = {Solución: },
    solution/print        = {true },
    % exercise/within       = section,
    % exercise/the-counter  = \thesection.\arabic{exercise},
  }

% END MODIFICATIONS

\begin{document}

\begin{exercise} Discutir el sistema y resolver en los casos compatibles
(2.5 puntos) \[
\begin{cases}
    2x + y + z  = a\\
    2x + y + 2z = 2a\\
    2x + y + 3z = 3
\end{cases}
\] \end{exercise}

\begin{exercise} Dada la matriz A, calcula: \[A = \begin{pmatrix}
1 & 0 & 0 & 1\\
2 & 3 & 1 & 4\\
1 & 6 & 2 & 4\\
\end{pmatrix}\]

\begin{enumerate}
\def\labelenumi{\alph{enumi})}
\item
  Su rango. (1.5 puntos)
\item
  Si existe, una columna combinación lineal de las restantes. (0.5
  puntos)
\item
  Si existe, una fila combinación lineal de las restantes. (0.5 puntos)
\end{enumerate}

\end{exercise}

\begin{exercise} Sea la matriz
\(A =  \begin{pmatrix}  x & 0 & 0 \\  2 & 3 & 1 \\  1 & 6 & 2 \\  \end{pmatrix} x \in \mathbb{R}\)

\begin{enumerate}
\def\labelenumi{\alph{enumi})}
\item
  Estudia para qué valores de x se cumple \(A^3 - I = O\) (\emph{I}
  matriz identidad, \emph{O} matriz nula). (1 punto)
\item
  Calcula \(A^12\) para los valores de \(x\) que verifican la condición
  anterior. (0.75 puntos)
\item
  Para \(x = 0\) y sabiendo que ese valor verifica la condición del
  primer apartado, calcula, si existe, la inversa de \emph{A}. (0.75
  puntos)
\end{enumerate}

\end{exercise}

\begin{exercise} Dado el sistema
\(\begin{cases} x + y + az = a\\ x + (a-1)y + az = 2\\ -x + z = 2 \end{cases}\)

\begin{enumerate}
\def\labelenumi{\alph{enumi})}
\tightlist
\item
  Estudia y clasifica el sistema según los valores de
  \(a \in \mathbb{R}\). (1.5 puntos)
\item
  Resuélvelo, si es posible, para el caso \(a = 2\). (1 punto)
\end{enumerate}

\end{exercise}

\begin{exercise} En una oficina se hicieron la semana pasada un total de
550 fotocopias entre fotocopias en blanco y negro y fotocopias en color.
El coste total de dichas fotocopias fue de 3,5 euros, siendo el coste de
cada fotocopia en blanco y negro de \emph{m} centimos de euro, y el
coste de cada fotocopia en color cuatro veces el coste de una en blanco
y negro.

\begin{enumerate}
\def\labelenumi{\alph{enumi})}
\tightlist
\item
  {[}0,5 puntos{]} Plantea un sistema de ecuaciones (en función de
  \emph{m}) donde las incógnitas \emph{x} e \emph{y} sean el número de
  fotocopias en blanco y negro y en color hechas la semana pasada.
\item
  {[}2 puntos{]} ¿Para qué valores de \emph{m} el sistema anterior tiene
  solución? En caso de existir solución, ¿es siempre ́unica? ¿Cuántas
  fotocopias en blanco y negro se realizaron enç la oficina si cada
  fotocopia en color costó 2 céntimos?
\end{enumerate}

\end{exercise}

\begin{exercise} En un local que se destinará a restaurante, se está
pensando en poner mesas altas y bajas. Las mesas altas necesitan una
superficie de 2 m\^{}2 cada una, mientras que las mesas bajas necesitan
una superficie de 4 m\^{}2 cada una. El local dedicará a mesas como
mucho una superficie de 120 m\^{}2. El propietario quiere que haya al
menos 5 mesas bajas y como mucho el doble de mesas altas que bajas.

\begin{enumerate}
\def\labelenumi{\alph{enumi})}
\tightlist
\item
  \textbf{{[}1,75 puntos{]}} ¿Cuántas mesas puede haber en el
  restaurante de cada tipo? Plantea el problema y representa
  gráficamente el conjunto de soluciones. ¿Podrá haber 15 mesas de cada
  tipo?
\item
  \textbf{{[}0,75 puntos{]}} Por estudios de mercado, se estima que el
  beneficio que dejan los clientes por mesa alta es de 20 euros,
  mientras que el beneficio por mesa baja es de 25 euros. ¿Cuántas mesas
  de cada tipo debe colocar para maximizar los beneficios estimados? ¿a
  cuánto ascenderían dichos beneficios?
\end{enumerate}

\end{exercise}

\begin{exercise} Sean las matrices
\(A = \begin{pmatrix}  m-1 & 0\\  -2 & m \\ \end{pmatrix}\),
\(B = \begin{pmatrix}  1 & -1\\  1 & 0 \\ \end{pmatrix}\),
\(C = \begin{pmatrix}  1 & -1\\  1 & 0 \\ \end{pmatrix}\) y
\(D = \begin{pmatrix}  1-2m\\  -2m \\ \end{pmatrix}\).

\begin{enumerate}
\def\labelenumi{\alph{enumi})}
\tightlist
\item
  \textbf{{[}1 puntos{]}} Si \((A+B)\cdot C = B \cdot D\), plantea un
  sistema de dos ecuaciones y dos incógnitas (representadas por \emph{x}
  e \emph{y}) en función del parámetro \emph{m}.
\item
  \textbf{{[}1,5 puntos{]}} ¿Para qué valores de \emph{m} el sistema
  anterior tiene solución? En caso de existir solución, ¿es siempre
  única? Resuelve el sistema para \(m = 2\).
\end{enumerate}

\end{exercise}

\begin{exercise} Una empresa puede contratar trabajadores de tipo A y
trabajadores de tipo B en una nueva factoría. Por convenio, es necesario
que haya mayor o igual número de trabajadores de tipo A que de tipo B y
que el número de trabajadores de tipo A no supere al doble del número de
trabajadores de tipo B. En total la empresa puede contratar un máximo de
30 trabajadores de tipo A y de 40 de tipo B.

\begin{enumerate}
\def\labelenumi{\alph{enumi})}
\tightlist
\item
  \textbf{{[}1,75 puntos{]}} ¿Cuántos trabajadores de cada tipo se
  pueden contratar en la empresa, de forma que se satisfagan todos los
  requisitos anteriores? Plantea el problema y representa gráficamente
  el conjunto de soluciones. ¿Podría contratarse a 20 trabajadores de
  tipo A y 15 de tipo B?
\item
  \textbf{{[}0,75 puntos{]}} Si el beneficio diario esperado para la
  empresa por cada trabajador de tipo A es de 240 euros y por cada
  trabajador de tipo B es de 200 euros, ¿cuántos trabajadores de cada
  tipo se deben contratar para maximizar el beneficio diario? ¿A cuánto
  asciende dicho beneficio máximo?
\end{enumerate}

\end{exercise}

\end{document}

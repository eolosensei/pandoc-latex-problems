\hypertarget{el-sistema-internacional-de-unidades}{%
\section{El Sistema Internacional de
Unidades}\label{el-sistema-internacional-de-unidades}}

\begin{exercise} Pon dos ejemplos de propiedades que sean magnitudes
físicas, y otros dos que no lo sean. Para cada una de las primeras,
enumera al menos tres unidades, una de ellas la del SI. \end{exercise}

\end{solution} Aaaa \end{solution}

\begin{exercise} Busca varios ejemplos de unidades que no pertenezcan al
SI. ¿Con qué magnitudes están relacionadas? Indica la equivalencia entre
estas unidades y las correspondientes del SI. \end{exercise}

\begin{exercise} ¿Crees que la yarda, definida en su día como unidad de
longitud y equivalente a 914~mm, y obtenida por la distancia marcada en
una vara entre la nariz y el dedo pulgar de la mano del rey Enrique I de
Inglaterra con su brazo estirado, sería hoy un procedimiento adecuado
para establecer una unidad de longitud?

\begin{enumerate}
\def\labelenumi{\alph{enumi})}
\tightlist
\item
  Apartado A
\item
  Apartado B
\end{enumerate}

\end{exercise}

\end{solution} Beeeeee \end{solution}

\begin{exercise} Las unidades del SI han sufrido cambios en su
definición a lo largo de la historia. Por ejemplo, el metro se definió
en 1790 como la diezmillonésima parte del cuadrante del meridiano
terrestre que pasa por París. En 1889 fue la distancia entre dos marcas
en una barra de aleación de platino-iridio que se guarda en Sévres. La
definición actual es de 1983. ¿A qué se deben estos cambios?
\end{exercise}

% Options for packages loaded elsewhere
\PassOptionsToPackage{unicode}{hyperref}
\PassOptionsToPackage{hyphens}{url}
%
\documentclass[
]{article}
\usepackage{lmodern}
\usepackage{amsmath}
\usepackage{ifxetex,ifluatex}
\ifnum 0\ifxetex 1\fi\ifluatex 1\fi=0 % if pdftex
  \usepackage[T1]{fontenc}
  \usepackage[utf8]{inputenc}
  \usepackage{textcomp} % provide euro and other symbols
  \usepackage{amssymb}
\else % if luatex or xetex
  \usepackage{unicode-math}
  \defaultfontfeatures{Scale=MatchLowercase}
  \defaultfontfeatures[\rmfamily]{Ligatures=TeX,Scale=1}
\fi
% Use upquote if available, for straight quotes in verbatim environments
\IfFileExists{upquote.sty}{\usepackage{upquote}}{}
\IfFileExists{microtype.sty}{% use microtype if available
  \usepackage[]{microtype}
  \UseMicrotypeSet[protrusion]{basicmath} % disable protrusion for tt fonts
}{}
\makeatletter
\@ifundefined{KOMAClassName}{% if non-KOMA class
  \IfFileExists{parskip.sty}{%
    \usepackage{parskip}
  }{% else
    \setlength{\parindent}{0pt}
    \setlength{\parskip}{6pt plus 2pt minus 1pt}}
}{% if KOMA class
  \KOMAoptions{parskip=half}}
\makeatother
\usepackage{xcolor}
\IfFileExists{xurl.sty}{\usepackage{xurl}}{} % add URL line breaks if available
\IfFileExists{bookmark.sty}{\usepackage{bookmark}}{\usepackage{hyperref}}
\hypersetup{
  hidelinks,
  pdfcreator={LaTeX via pandoc}}
\urlstyle{same} % disable monospaced font for URLs
\setlength{\emergencystretch}{3em} % prevent overfull lines
\providecommand{\tightlist}{%
  \setlength{\itemsep}{0pt}\setlength{\parskip}{0pt}}
\setcounter{secnumdepth}{-\maxdimen} % remove section numbering
\ifluatex
  \usepackage{selnolig}  % disable illegal ligatures
\fi

\author{}
\date{}

\begin{document}

\hypertarget{el-sistema-internacional-de-unidades}{%
\section{El Sistema Internacional de
Unidades}\label{el-sistema-internacional-de-unidades}}

\begin{exercise}Pon dos ejemplos de propiedades que sean magnitudes
físicas, y otros dos que no lo sean. Para cada una de las primeras,
enumera al menos tres unidades, una de ellas la del SI.\end{exercise}

\begin{solution}Aaaaaa\end{solution}

\begin{exercise}Busca varios ejemplos de unidades que no pertenezcan al
SI. ¿Con qué magnitudes están relacionadas? Indica la equivalencia entre
estas unidades y las correspondientes del SI.

Por otro lado, meto compuestos químicos como el AgCl, el \ch{Mg(OH)3} y
el \ch{CH3-CH2-OH}, así como el \ch{Cu(NO3)2}. Y aquí viene uno con
paréntesis y punto: \ch{(C6H12O6}).\end{exercise}

\begin{exercise}[OXF15] ¿Crees que la yarda, definida en su día como
unidad de longitud y equivalente a 914~mm, y obtenida por la distancia
marcada en una vara entre la nariz y el dedo pulgar de la mano del rey
Enrique I de Inglaterra con su brazo estirado, sería hoy un
procedimiento adecuado para establecer una unidad de
longitud?\end{exercise}

\begin{solution}Beeeeee\end{solution}

\begin{exercise}Las unidades del SI han sufrido cambios en su definición
a lo largo de la historia. Por ejemplo, el metro se definió en 1790 como
la diezmillonésima parte del cuadrante del meridiano terrestre que pasa
por París.

En 1889 fue la distancia entre dos marcas en una barra de aleación de
platino-iridio que se guarda en Sévres. La definición actual es de 1983.
¿A qué se deben estos cambios?\end{exercise}

\begin{solution}Beeeeee\end{solution}

\begin{exercise}Aquí voy a poner un listado de unidades, a ver qué pasa:
5~mm, 43 km, 83 m/s.

\begin{enumerate}
\def\labelenumi{\alph{enumi})}
\tightlist
\item
  Uno
\item
  Dos
\item
  Tres
\end{enumerate}

\end{exercise}

\begin{solution}Beeeeee\end{solution}

\begin{exercise}

\begin{enumerate}
\def\labelenumi{\alph{enumi})}
\tightlist
\item
  Uno
\item
  Dos
\item
  Tres
\end{enumerate}

\end{exercise}

\begin{solution}Primera linea de la solución

Segunda línea de la solución\end{solution}

\end{document}
